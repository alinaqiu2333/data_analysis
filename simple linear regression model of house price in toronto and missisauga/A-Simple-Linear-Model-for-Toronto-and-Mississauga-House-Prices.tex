% Options for packages loaded elsewhere
\PassOptionsToPackage{unicode}{hyperref}
\PassOptionsToPackage{hyphens}{url}
%
\documentclass[
]{article}
\usepackage{lmodern}
\usepackage{amssymb,amsmath}
\usepackage{ifxetex,ifluatex}
\ifnum 0\ifxetex 1\fi\ifluatex 1\fi=0 % if pdftex
  \usepackage[T1]{fontenc}
  \usepackage[utf8]{inputenc}
  \usepackage{textcomp} % provide euro and other symbols
\else % if luatex or xetex
  \usepackage{unicode-math}
  \defaultfontfeatures{Scale=MatchLowercase}
  \defaultfontfeatures[\rmfamily]{Ligatures=TeX,Scale=1}
\fi
% Use upquote if available, for straight quotes in verbatim environments
\IfFileExists{upquote.sty}{\usepackage{upquote}}{}
\IfFileExists{microtype.sty}{% use microtype if available
  \usepackage[]{microtype}
  \UseMicrotypeSet[protrusion]{basicmath} % disable protrusion for tt fonts
}{}
\makeatletter
\@ifundefined{KOMAClassName}{% if non-KOMA class
  \IfFileExists{parskip.sty}{%
    \usepackage{parskip}
  }{% else
    \setlength{\parindent}{0pt}
    \setlength{\parskip}{6pt plus 2pt minus 1pt}}
}{% if KOMA class
  \KOMAoptions{parskip=half}}
\makeatother
\usepackage{xcolor}
\IfFileExists{xurl.sty}{\usepackage{xurl}}{} % add URL line breaks if available
\IfFileExists{bookmark.sty}{\usepackage{bookmark}}{\usepackage{hyperref}}
\hypersetup{
  pdftitle={A Simple Linear Model for Toronto and Mississauga House Prices},
  pdfauthor={AQ3383},
  hidelinks,
  pdfcreator={LaTeX via pandoc}}
\urlstyle{same} % disable monospaced font for URLs
\usepackage[margin=1in]{geometry}
\usepackage{color}
\usepackage{fancyvrb}
\newcommand{\VerbBar}{|}
\newcommand{\VERB}{\Verb[commandchars=\\\{\}]}
\DefineVerbatimEnvironment{Highlighting}{Verbatim}{commandchars=\\\{\}}
% Add ',fontsize=\small' for more characters per line
\usepackage{framed}
\definecolor{shadecolor}{RGB}{248,248,248}
\newenvironment{Shaded}{\begin{snugshade}}{\end{snugshade}}
\newcommand{\AlertTok}[1]{\textcolor[rgb]{0.94,0.16,0.16}{#1}}
\newcommand{\AnnotationTok}[1]{\textcolor[rgb]{0.56,0.35,0.01}{\textbf{\textit{#1}}}}
\newcommand{\AttributeTok}[1]{\textcolor[rgb]{0.77,0.63,0.00}{#1}}
\newcommand{\BaseNTok}[1]{\textcolor[rgb]{0.00,0.00,0.81}{#1}}
\newcommand{\BuiltInTok}[1]{#1}
\newcommand{\CharTok}[1]{\textcolor[rgb]{0.31,0.60,0.02}{#1}}
\newcommand{\CommentTok}[1]{\textcolor[rgb]{0.56,0.35,0.01}{\textit{#1}}}
\newcommand{\CommentVarTok}[1]{\textcolor[rgb]{0.56,0.35,0.01}{\textbf{\textit{#1}}}}
\newcommand{\ConstantTok}[1]{\textcolor[rgb]{0.00,0.00,0.00}{#1}}
\newcommand{\ControlFlowTok}[1]{\textcolor[rgb]{0.13,0.29,0.53}{\textbf{#1}}}
\newcommand{\DataTypeTok}[1]{\textcolor[rgb]{0.13,0.29,0.53}{#1}}
\newcommand{\DecValTok}[1]{\textcolor[rgb]{0.00,0.00,0.81}{#1}}
\newcommand{\DocumentationTok}[1]{\textcolor[rgb]{0.56,0.35,0.01}{\textbf{\textit{#1}}}}
\newcommand{\ErrorTok}[1]{\textcolor[rgb]{0.64,0.00,0.00}{\textbf{#1}}}
\newcommand{\ExtensionTok}[1]{#1}
\newcommand{\FloatTok}[1]{\textcolor[rgb]{0.00,0.00,0.81}{#1}}
\newcommand{\FunctionTok}[1]{\textcolor[rgb]{0.00,0.00,0.00}{#1}}
\newcommand{\ImportTok}[1]{#1}
\newcommand{\InformationTok}[1]{\textcolor[rgb]{0.56,0.35,0.01}{\textbf{\textit{#1}}}}
\newcommand{\KeywordTok}[1]{\textcolor[rgb]{0.13,0.29,0.53}{\textbf{#1}}}
\newcommand{\NormalTok}[1]{#1}
\newcommand{\OperatorTok}[1]{\textcolor[rgb]{0.81,0.36,0.00}{\textbf{#1}}}
\newcommand{\OtherTok}[1]{\textcolor[rgb]{0.56,0.35,0.01}{#1}}
\newcommand{\PreprocessorTok}[1]{\textcolor[rgb]{0.56,0.35,0.01}{\textit{#1}}}
\newcommand{\RegionMarkerTok}[1]{#1}
\newcommand{\SpecialCharTok}[1]{\textcolor[rgb]{0.00,0.00,0.00}{#1}}
\newcommand{\SpecialStringTok}[1]{\textcolor[rgb]{0.31,0.60,0.02}{#1}}
\newcommand{\StringTok}[1]{\textcolor[rgb]{0.31,0.60,0.02}{#1}}
\newcommand{\VariableTok}[1]{\textcolor[rgb]{0.00,0.00,0.00}{#1}}
\newcommand{\VerbatimStringTok}[1]{\textcolor[rgb]{0.31,0.60,0.02}{#1}}
\newcommand{\WarningTok}[1]{\textcolor[rgb]{0.56,0.35,0.01}{\textbf{\textit{#1}}}}
\usepackage{longtable,booktabs}
% Correct order of tables after \paragraph or \subparagraph
\usepackage{etoolbox}
\makeatletter
\patchcmd\longtable{\par}{\if@noskipsec\mbox{}\fi\par}{}{}
\makeatother
% Allow footnotes in longtable head/foot
\IfFileExists{footnotehyper.sty}{\usepackage{footnotehyper}}{\usepackage{footnote}}
\makesavenoteenv{longtable}
\usepackage{graphicx,grffile}
\makeatletter
\def\maxwidth{\ifdim\Gin@nat@width>\linewidth\linewidth\else\Gin@nat@width\fi}
\def\maxheight{\ifdim\Gin@nat@height>\textheight\textheight\else\Gin@nat@height\fi}
\makeatother
% Scale images if necessary, so that they will not overflow the page
% margins by default, and it is still possible to overwrite the defaults
% using explicit options in \includegraphics[width, height, ...]{}
\setkeys{Gin}{width=\maxwidth,height=\maxheight,keepaspectratio}
% Set default figure placement to htbp
\makeatletter
\def\fps@figure{htbp}
\makeatother
\setlength{\emergencystretch}{3em} % prevent overfull lines
\providecommand{\tightlist}{%
  \setlength{\itemsep}{0pt}\setlength{\parskip}{0pt}}
\setcounter{secnumdepth}{-\maxdimen} % remove section numbering

\title{A Simple Linear Model for Toronto and Mississauga House Prices}
\author{AQ3383}
\date{October 20, 2020}

\begin{document}
\maketitle

\hypertarget{i.-exploratory-data-analysis}{%
\subsection{I. Exploratory Data
Analysis}\label{i.-exploratory-data-analysis}}

\begin{Shaded}
\begin{Highlighting}[]
\KeywordTok{set.seed}\NormalTok{(}\DecValTok{3383}\NormalTok{)}
\NormalTok{data_AQ3383 =}\StringTok{ }\NormalTok{original_data_AQ3383[}\KeywordTok{sample}\NormalTok{(}\KeywordTok{nrow}\NormalTok{(original_data_AQ3383), }\DecValTok{200}\NormalTok{), ]}
\end{Highlighting}
\end{Shaded}

Part 1 The boxplot with no unusual points removed is shown below.
\includegraphics{A-Simple-Linear-Model-for-Toronto-and-Mississauga-House-Prices_files/figure-latex/unnamed-chunk-3-1.pdf}

I use this plot because we can easily see if any points are outliers
from graph plotted. In our case, the outlier is the maximum value.
Therefore when creating a subset of what we currently have, I remove
this particular point.

\begin{Shaded}
\begin{Highlighting}[]
\CommentTok{#find the ID of the outlier}
\NormalTok{data_AQ3383[}\KeywordTok{which.max}\NormalTok{(data_AQ3383}\OperatorTok{$}\NormalTok{list),]}
\end{Highlighting}
\end{Shaded}

\begin{verbatim}
##      ID  sold  list taxes location
## 112 112 1.085 84.99  4457        T
\end{verbatim}

\begin{Shaded}
\begin{Highlighting}[]
\CommentTok{#remove the outlier}
\NormalTok{df_AQ3383 <-}\StringTok{ }\NormalTok{data_AQ3383[data_AQ3383}\OperatorTok{$}\NormalTok{ID }\OperatorTok{!=}\StringTok{ }\DecValTok{112}\NormalTok{,]}
\end{Highlighting}
\end{Shaded}

Then we use data frame df\_AQ3383 for the rest of the assignment.
\includegraphics{A-Simple-Linear-Model-for-Toronto-and-Mississauga-House-Prices_files/figure-latex/unnamed-chunk-5-1.pdf}
\includegraphics{A-Simple-Linear-Model-for-Toronto-and-Mississauga-House-Prices_files/figure-latex/unnamed-chunk-5-2.pdf}

Interpretation: Both graphs above shows a somehow linear relationship
between, therefore we can use linear regression for both graphs. Sold
price vs list price has a more significant linear relationship than sold
price vs tax price. Heteroscedasticity occurs in sold price vs tax price
graph. In the boxplot with all data of listing price included, there
exists a significant outlier, which is about 85 million CAD. This might
be a error occurred when measuring data.

\hypertarget{ii.-methods-and-model}{%
\subsection{II. Methods and Model}\label{ii.-methods-and-model}}

Three simple linear regressions (SLR) for sale price from list price and
its corresponding graph shown below:

\begin{verbatim}
## 
## Call:
## lm(formula = dfy_AQ3383 ~ dfx1_AQ3383)
## 
## Residuals:
##     Min      1Q  Median      3Q     Max 
## -4.9821 -0.1073 -0.0242  0.1184  0.7429 
## 
## Coefficients:
##             Estimate Std. Error t value Pr(>|t|)    
## (Intercept)  0.35438    0.05900   6.006 9.01e-09 ***
## dfx1_AQ3383  0.77949    0.02795  27.887  < 2e-16 ***
## ---
## Signif. codes:  0 '***' 0.001 '**' 0.01 '*' 0.05 '.' 0.1 ' ' 1
## 
## Residual standard error: 0.4173 on 197 degrees of freedom
## Multiple R-squared:  0.7979, Adjusted R-squared:  0.7969 
## F-statistic: 777.7 on 1 and 197 DF,  p-value: < 2.2e-16
\end{verbatim}

\begin{verbatim}
## 
## Call:
## lm(formula = dfy_M_AQ3383 ~ dfx_M_AQ3383)
## 
## Residuals:
##      Min       1Q   Median       3Q      Max 
## -0.45148 -0.04358 -0.01940  0.05841  0.41395 
## 
## Coefficients:
##              Estimate Std. Error t value Pr(>|t|)    
## (Intercept)   0.13901    0.02145   6.481 5.48e-09 ***
## dfx_M_AQ3383  0.89045    0.01188  74.941  < 2e-16 ***
## ---
## Signif. codes:  0 '***' 0.001 '**' 0.01 '*' 0.05 '.' 0.1 ' ' 1
## 
## Residual standard error: 0.1031 on 86 degrees of freedom
## Multiple R-squared:  0.9849, Adjusted R-squared:  0.9847 
## F-statistic:  5616 on 1 and 86 DF,  p-value: < 2.2e-16
\end{verbatim}

\begin{verbatim}
## 
## Call:
## lm(formula = dfy_T_AQ3383 ~ dfx_T_AQ3383)
## 
## Residuals:
##     Min      1Q  Median      3Q     Max 
## -4.6577 -0.1544 -0.0038  0.1853  0.7889 
## 
## Coefficients:
##              Estimate Std. Error t value Pr(>|t|)    
## (Intercept)   0.54309    0.10714   5.069 1.65e-06 ***
## dfx_T_AQ3383  0.70401    0.04609  15.276  < 2e-16 ***
## ---
## Signif. codes:  0 '***' 0.001 '**' 0.01 '*' 0.05 '.' 0.1 ' ' 1
## 
## Residual standard error: 0.5367 on 109 degrees of freedom
## Multiple R-squared:  0.6816, Adjusted R-squared:  0.6787 
## F-statistic: 233.4 on 1 and 109 DF,  p-value: < 2.2e-16
\end{verbatim}

Table:

\begin{longtable}[]{@{}lcccccc@{}}
\toprule
\begin{minipage}[b]{0.10\columnwidth}\raggedright
Regression\strut
\end{minipage} & \begin{minipage}[b]{0.12\columnwidth}\centering
\(R^2\)\strut
\end{minipage} & \begin{minipage}[b]{0.12\columnwidth}\centering
estimated intercept \(\beta_0\)\strut
\end{minipage} & \begin{minipage}[b]{0.12\columnwidth}\centering
estimated slope \(\beta_1\)\strut
\end{minipage} & \begin{minipage}[b]{0.12\columnwidth}\centering
estimate of the variance of the error\strut
\end{minipage} & \begin{minipage}[b]{0.12\columnwidth}\centering
p-value for \(H_0: \beta_1=0\)\strut
\end{minipage} & \begin{minipage}[b]{0.12\columnwidth}\centering
95\% CI for \(\beta_1\)\strut
\end{minipage}\tabularnewline
\midrule
\endhead
\begin{minipage}[t]{0.10\columnwidth}\raggedright
All\strut
\end{minipage} & \begin{minipage}[t]{0.12\columnwidth}\centering
0.7979\strut
\end{minipage} & \begin{minipage}[t]{0.12\columnwidth}\centering
0.3544\strut
\end{minipage} & \begin{minipage}[t]{0.12\columnwidth}\centering
0.7795\strut
\end{minipage} & \begin{minipage}[t]{0.12\columnwidth}\centering
0.1741\strut
\end{minipage} & \begin{minipage}[t]{0.12\columnwidth}\centering
p-value: \textless{} 2.2e-16\strut
\end{minipage} & \begin{minipage}[t]{0.12\columnwidth}\centering
(0.7244, 0.8346)\strut
\end{minipage}\tabularnewline
\begin{minipage}[t]{0.10\columnwidth}\raggedright
Mississauga Neighborhood\strut
\end{minipage} & \begin{minipage}[t]{0.12\columnwidth}\centering
0.9849\strut
\end{minipage} & \begin{minipage}[t]{0.12\columnwidth}\centering
0.1390\strut
\end{minipage} & \begin{minipage}[t]{0.12\columnwidth}\centering
0.8905\strut
\end{minipage} & \begin{minipage}[t]{0.12\columnwidth}\centering
0.0106\strut
\end{minipage} & \begin{minipage}[t]{0.12\columnwidth}\centering
p-value: \textless{} 2.2e-16\strut
\end{minipage} & \begin{minipage}[t]{0.12\columnwidth}\centering
(0.8668, 0.9141)\strut
\end{minipage}\tabularnewline
\begin{minipage}[t]{0.10\columnwidth}\raggedright
Toronto Neighborhood\strut
\end{minipage} & \begin{minipage}[t]{0.12\columnwidth}\centering
0.6816\strut
\end{minipage} & \begin{minipage}[t]{0.12\columnwidth}\centering
0.5431\strut
\end{minipage} & \begin{minipage}[t]{0.12\columnwidth}\centering
0.7040\strut
\end{minipage} & \begin{minipage}[t]{0.12\columnwidth}\centering
0.2880\strut
\end{minipage} & \begin{minipage}[t]{0.12\columnwidth}\centering
p-value: \textless{} 2.2e-16\strut
\end{minipage} & \begin{minipage}[t]{0.12\columnwidth}\centering
(0.6127, 0.7954)\strut
\end{minipage}\tabularnewline
\bottomrule
\end{longtable}

Interpret and compare: We see the difference between \(R^2\) is quite
different. Mississauga neighborhood has a 0.98 \(R^2\), which is the
highest among all. However Toronto Neighborhood has a 0.68 \(R^2\). As a
result, \(R^2\) based on all data is 0.80, which is in between other two
sets of data. This shows us it is necessary to evaluate two
neighborhoods respectively. This is normal because the distance between
data and fitted regression line for all neighborhoods is usually not as
well as how data fits the two neighborhoods respectively, since the list
and sold price are often more similar in one neighborhoods.

A pooled two-sample t-test is not the best to be used when determine if
there is a statistically significant difference between the slopes of
the simple linear models for the two neighborhoods. Since we are dealing
with housing price in two different cities, therefore it is reasonable
to assume the two sets of data are independent. However, according to
data shown above, they do not have the same variance. Therefore we
should not use pooled two-sample t-test here.

\hypertarget{iii.-discussions-and-limitations}{%
\subsection{III. Discussions and
Limitations}\label{iii.-discussions-and-limitations}}

\includegraphics{A-Simple-Linear-Model-for-Toronto-and-Mississauga-House-Prices_files/figure-latex/unnamed-chunk-7-1.pdf}

According to data summary shown above, I picked Mississauga neighborhood
to do the following evaluation. This is because it has a \(R^2\) of
0.9849, which means it has the highest portion of explained response
variable variation by this model.

Violations: According to fitted vs residual graph, we can tell there are
a few irregulars. The graph is mostly linear, which is good. There are
also a few outliers in Normal Q-Q plot. However thee residuals follow a
mostly straight line, which concluded there are not much violations,
except for a few outliers.

Possible predictors to to fit a multiple linear regression for sale
price could be size of the house, and how old this house is. These are
factors directly related and can cause effect to the sale price

\end{document}
